\documentclass[aspectratio=169]{beamer}

% OPTION 1: Singapore theme with beaver color scheme
% \usetheme{Singapore}
% \usecolortheme{beaver}
% \useinnertheme{circles}
% \useoutertheme{miniframes}

% Uncomment one of the other options below if you prefer

% OPTION 2: Metropolis theme (modern, minimalist)
% \usetheme{metropolis}
% \setbeamercolor{background canvas}{bg=white}
% \setbeamercolor{frametitle}{bg=gray!10}

% OPTION 3: Frankfurt with spruce color scheme
% \usetheme{Frankfurt}
% \usecolortheme{spruce}
% \useinnertheme{circles}

% OPTION 4: Dresden with orchid color scheme
% \usetheme{Madrid}
% % \usecolortheme{beaver}
% \useinnertheme{rectangles}
% \useoutertheme{miniframes}

% \documentclass{beamer}
\usepackage{graphicx} % Required for inserting images
\usepackage{listings}
\usepackage{amsmath}
\usepackage{animate}
\usetheme{Berkeley}
\definecolor{mycolor}{RGB}{120, 130, 170} 
\setbeamercolor{structure}{fg=mycolor}
\setbeamercolor{background canvas}{bg=white!90} 
\usepackage{tikz}
\usetikzlibrary{trees}


\usetikzlibrary{shapes, arrows}

\tikzstyle{block} = [rectangle, draw, fill=white!20, text width=6em, text centered, rounded corners, minimum height=4em]
\tikzstyle{line} = [draw, -latex']
\tikzstyle{cloud} = [draw, ellipse, fill=white!20, minimum height=2em]


% \title{Statistical Estimation Methods}

% \subtitle{A Brief Overview}
% \institute{Department of Mathematics, IIT Roorkee}
% \date{6 April 2023}
% \author[]{Ved Umrajkar}

\logo{\includegraphics[height=1cm]{iitr-logo.jpg}}


% Required packages
\usepackage{amsmath, amssymb, amsthm}
\usepackage{tikz}
\usepackage{algorithmic}
\usepackage{algorithm}
\usepackage{mathtools}
\usepackage{listings}
\usepackage{xcolor}

% Code listings setup
\lstset{
  language=Python,
  basicstyle=\footnotesize\ttfamily,
  keywordstyle=\color{blue},
  commentstyle=\color{green!60!black},
  stringstyle=\color{red},
  numbers=left,
  numberstyle=\tiny\color{gray},
  stepnumber=1,
  numbersep=10pt,
  backgroundcolor=\color{white},
  tabsize=4,
  showspaces=false,
  showstringspaces=false,
  frame=single,
  breaklines=true,
  breakatwhitespace=true,
}

% Title information
\title[Discrete Logarithm Problem]{Hands-On Discrete Logarithm Problems}
\subtitle{Algorithms and Cryptographic Applications}
\author[]{Ved Umrajkar}
\institute[]{Department of Mathematics\\Indian Institute of Technology, Roorkee}
\date{\today}

\begin{document}

% Title slide
\begin{frame}
  \titlepage
\end{frame}

% Outline slide
\begin{frame}{Outline}
  \tableofcontents
\end{frame}

% Start presentation content
\section{Introduction to Discrete Logarithms}

\begin{frame}{What is a Discrete Logarithm?}
  \begin{block}{Definition}
    In a finite group $G$, the \alert{discrete logarithm problem} is:
    
    Given elements $\alpha, \beta \in G$ with $\beta \in \langle\alpha\rangle$, find the least positive integer $x$ such that $\alpha^x = \beta$.
  \end{block}
  
  \begin{itemize}
    \item We denote this value $x$ as $\log_\alpha \beta$
    \item In additive notation: find $x$ such that $x\alpha = \beta$
    \item Multiplicative vs. additive notation (depending on group context)
  \end{itemize}
\end{frame}

\begin{frame}{Group Context Matters}
  \begin{columns}
    \begin{column}{0.48\textwidth}
      \textbf{Example in $\mathbb{F}_{101}^{\times}$:}
      \begin{itemize}
        \item $\log_3 37 = 24$ (multiplicative)
        \item Because $3^{24} \equiv 37 \pmod{101}$
      \end{itemize}
    \end{column}
    \begin{column}{0.48\textwidth}
      \textbf{Example in $\mathbb{F}_{101}^{+}$:}
      \begin{itemize}
        \item $\log_3 37 = 46$ (additive)
        \item Because $46 \cdot 3 \equiv 37 \pmod{101}$
      \end{itemize}
    \end{column}
  \end{columns}
  \vspace{0.5cm}
  \textbf{Key insight:} The difficulty depends on how the group is represented, not just its structure.
\end{frame}

\begin{frame}{Variants of the Discrete Logarithm Problem}
  \begin{enumerate}
    \item \textbf{Standard DLP:} Given $\alpha, \beta \in G$ with $\beta \in \langle\alpha\rangle$, find $x$ such that $\alpha^x = \beta$
    
    \item \textbf{Decision problem:} Given $\alpha, \beta \in G$, determine if $\beta \in \langle\alpha\rangle$
    
    \item \textbf{Extended DLP:} Given $\alpha, \beta \in G$, find the least positive integers $(x,y)$ such that $\beta^y = \alpha^x$
    
    \item \textbf{Vector DLP:} Given $\alpha_1, \ldots, \alpha_r \in G$ and $\beta \in G$, find $(e_1, \ldots, e_r)$ such that $\beta = \alpha_1^{e_1} \cdots \alpha_r^{e_r}$
  \end{enumerate}
\end{frame}

\begin{frame}{Why Discrete Logarithms Matter}
  \begin{block}{Cryptographic Significance}
    The discrete logarithm problem forms the foundation of many cryptographic protocols:
  \end{block}
  
  \begin{itemize}
    \item \textbf{Diffie-Hellman key exchange}
    \item \textbf{ElGamal encryption system}
    \item \textbf{Digital Signature Algorithm (DSA)}
    \item \textbf{Elliptic Curve Cryptography (ECC)}
  \end{itemize}
  
  \textbf{Security principle:} Easy in one direction (exponentiation), hard in the other (logarithm)
\end{frame}

\begin{frame}{Primitive Roots and Discrete Logarithms}
  \begin{block}{Primitive Roots}
    An element $u$ is a \alert{primitive root} modulo $m$ if the order of $u$ is $\phi(m)$.
  \end{block}
  
  \begin{itemize}
    \item If $u$ is a primitive root modulo $m$, then every unit modulo $m$ can be written as a power of $u$
    \item Example: $2$ is a primitive root modulo $11$ since its powers generate all units $\{1,2,3,...,10\}$
    \item Primitive roots exist modulo $p$ (prime) and for certain other moduli
  \end{itemize}
  
  \begin{block}{Computing Discrete Logarithms}
    For a primitive root $g$ modulo $p$, computing $\log_g a$ means finding the isomorphism between $\langle g \rangle$ and $\mathbb{Z}/p\mathbb{Z}$
  \end{block}
\end{frame}

\begin{frame}{Computation: Easy vs. Hard Cases}
  \begin{columns}
    \begin{column}{0.48\textwidth}
      \textbf{Easy cases:}
      \begin{itemize}
        \item Additive groups of finite fields
        \item Groups with smooth order (factorizable with small primes)
        \item Small groups (brute force)
      \end{itemize}
    \end{column}
    \begin{column}{0.48\textwidth}
      \textbf{Hard cases:}
      \begin{itemize}
        \item Multiplicative groups of finite fields with large prime factors
        \item Elliptic curve groups over finite fields
        \item Generic groups (black-box access only)
      \end{itemize}
    \end{column}
  \end{columns}
  
  \vspace{0.5cm}
  \begin{block}{Computational Hardness}
    The security of many cryptosystems relies on the computational intractability of the discrete logarithm problem in specific groups.
  \end{block}
\end{frame}

\begin{frame}{A Simple Example}
  Let's compute discrete logarithms in a small group:
  
  \begin{block}{Example: Finding all discrete logarithms base $2$ modulo $11$}
    \begin{center}
      \begin{tabular}{c|ccccccccc}
        $n$ & 1 & 2 & 3 & 4 & 5 & 6 & 7 & 8 & 9 & 10 \\
        \hline
        $\log_2 n$ & 0 & 1 & 8 & 2 & 4 & 9 & 7 & 3 & 6 & 5 \\
      \end{tabular}
    \end{center}
  \end{block}
  
  \begin{itemize}
    \item Verify: $2^8 \equiv 3 \pmod{11}$, $2^9 \equiv 6 \pmod{11}$, etc.
    \item With this table, multiplication becomes addition of logarithms!
    \item $3 \cdot 6 \equiv 7 \pmod{11}$ and $\log_2(3) + \log_2(6) \equiv \log_2(7) \pmod{10}$
  \end{itemize}
\end{frame}

\begin{frame}{A Computational Perspective}
  \begin{block}{Computing Discrete Logarithms in Practice}
    Several categories of algorithms:
  \end{block}
  
  \begin{enumerate}
    \item \textbf{Exhaustive search}: $O(N)$ where $N$ is the group order
    \item \textbf{Baby-step Giant-step}: $O(\sqrt{N})$ time and space
    \item \textbf{Pohlig-Hellman}: Reduces the problem for composite orders
    \item \textbf{Pollard's $\rho$ method}: $O(\sqrt{N})$ time, $O(1)$ space
    \item \textbf{Index Calculus}: Sub-exponential time for certain groups
  \end{enumerate}
  
  \begin{alertblock}{Hard Problem vs. Trapdoor}
    Unlike RSA, discrete logarithm-based systems typically don't have "trapdoors" that make computation easy for the legitimate user.
  \end{alertblock}
\end{frame}

\end{document}